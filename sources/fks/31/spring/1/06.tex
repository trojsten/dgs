Vladko sa na hodine fyziky dozvedel, že ak zoberieme vodič a vytvarujeme
ho do uzavretej slučky, tak pokiaľ sa bude meniť tok magnetického poľa
prechádzajúci touto slučkou, potom sme schopní odmerať multimetrom
indukované elektromotorické napätie na tejto slučke. Vladko je však
špekulant a vymyslel nasledujúci pokus:

Zobral cievku s plochou prierezu $\SI{20}{\centi\metre\squared}$, v
ktorej sa zvyšuje magnetické pole rýchlosťou
$\SI{2}{\milli\tesla\per\second}$. Následne zobral drôt s celkovým
odporom $\SI{100}{\ohm}$ a raz ho obtočil okolo cievky tak, že drôt
vytvoril uzavretú slučku. Následne na dve miesta v slučke pripojil dva
rovnaké multimetre s obrovským vnútorným odporom podľa obrázka a začal
merať indukované napätie na slučke.

Od vás by chcel vedieť, aké napätia ukážu jednotlivé multimetre. Nie je
tento výsledok v rozpore s tým, čo sme sa učili v škole o napätí pri
elektrických obvodoch? Ako je potom možné, že multimetre zobrazia takéto
hodnoty?

{[}OBRAZOK{]}
