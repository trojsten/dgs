%%%%%%%%%%%%%%%%%%%%%%%%%%%%%%% Operators %%%%%%%%%%%%%%%%%%%%%%%%%%%%%%%%%%%%%%%%%%

\ExplSyntaxOn
\makeatletter

\NewDocumentCommand{\cdiff}{O{,} m m}{\dgs_split_diff:nnn {#1} {#2} {#3}}

\cs_new_protected:Npn \dgs_split_diff:nnn #1 #2 #3 {
    \seq_set_split:Nnn \l_split_args_seq{#1}{#2}
    \seq_map_inline:Nn \l_split_args_seq {#3 ##1}
}

\NewDocumentCommand{\Tuple}{O{;} >{\SplitList{;}} m}{%
    \def\itemdelim@{\def\itemdelim@{#1}}%
    {%
        \left(\ProcessList{#2}{\insert@separator}\right)%
    }%
}

\NewDocumentCommand{\Coord}{O{;} >{\SplitList{;}} m}{%
    \def\itemdelim@{\def\itemdelim@{#1}}%
    {%
        \left[\ProcessList{#2}{\insert@separator}\right]%
    }%
}

\newcommand\insert@separator[1]{\itemdelim@ #1}
\seq_new:N \l_split_args_seq

\ExplSyntaxOff

\DeclareMathOperator{\arccot}{\mathrm{arccot}}

% Section: differentials

% Regular upright differential
\NewDocumentCommand{\diff}{}{\mathop{}\!\mathrm{d}}
% Partial differential
\NewDocumentCommand{\pdiff}{}{\mathop{}\!\partial}
% Finite differential
\NewDocumentCommand{\fdiff}{}{\mathop{}\!\Delta}
% Small delta differential
\NewDocumentCommand{\udiff}{}{\mathop{}\!\delta}

% Power differential
\NewDocumentCommand{\Diff}{o}{%
    \IfNoValueTF{#1}{\diff}{\diff^{#1}}%
}
\NewDocumentCommand{\PDiff}{o}{%
    \IfNoValueTF{#1}{\pdiff}{\pdiff^{#1}}%
}

% Inner derivative with any differential symbol

\NewDocumentCommand{\frac@}{}{}

\ExplSyntaxOn
\NewDocumentCommand{\SetDerivativeFrac@}{m}{
    \renewcommand{\frac@}{\str_case:nnF{#1}{%
            {d}{\dfrac}%
            {t}{\tfrac}%
            {s}{\sfrac}%
            {n}{\nicefrac}%
    }{\frac}}%
}
\ExplSyntaxOff

\NewDocumentCommand{\Derivative@}{d<> o D<>{\diff} m m}{%
    \SetDerivativeFrac@{#1}
    \IfNoValueTF{#2}{%
        \frac@{#3#4}{\cdiff{#5}{#3}}%
    }{%
        \frac@{#3^{#2}#4}{\cdiff{#5}{#3}}%
    }%
}

\NewDocumentCommand{\DerivativeParentheses@}{d<> o D<>{\diff} m m}{%
    \SetDerivativeFrac@{#1}
    \IfNoValueTF{#2}{%
        \frac@{#3}{#3#5}\left(#4\right)%
    }{%
        \frac@{#3^{#2}}{#3{#5}^{#2}}\left(#4\right)%
    }%
}

\ExplSyntaxOn
\NewDocumentCommand{\DerivativeEmpty@}{d<> o D<>{\diff} m}{%
    \renewcommand{\frac@}{\str_case:nnF{#1}{%
            {d}{\dfrac}%
            {t}{\tfrac}%
            {s}{\sfrac}%
            {n}{\nicefrac}%
    }{\frac}}%
    \IfNoValueTF{#2}{%
        \frac@{#3}{#3#4}%
    }{%
        \frac@{#3^{#2}}{#3#4^{#2}}%
    }%
}
\ExplSyntaxOff

% \Derivative[order]{what}{by-what} + shorthands for Partial, Finite and Delta
\NewDocumentCommand{\Derivative}{d<> o m m}{\Derivative@<#1>[#2]<\diff>{#3}{#4}}
\NewDocumentCommand{\PDerivative}{d<> o m m}{\Derivative@<#1>[#2]<\pdiff>{#3}{#4}}
\NewDocumentCommand{\FDerivative}{d<> o m m}{\Derivative@<#1>[#2]<\fdiff>{#3}{#4}}
\NewDocumentCommand{\UDerivative}{d<> o m m}{\Derivative@<#1>[#2]<\udiff>{#3}{#4}}
\newcommand*\Drv{\Derivative}
\newcommand*\PDrv{\PDerivative}
\newcommand*\FDrv{\FDerivative}
\newcommand*\UDrv{\UDerivative}

% \DerivativeEmpty[order]{by-what} + shorthands for Partial, Finite and Delta
\NewDocumentCommand{\DerivativeEmpty}{d<> o m}{\DerivativeEmpty@<#1>[#2]<\diff>{#3}}
\NewDocumentCommand{\PDerivativeEmpty}{d<> o m}{\DerivativeEmpty@<#1>[#2]<\pdiff>{#3}}
\NewDocumentCommand{\FDerivativeEmpty}{d<> o m}{\DerivativeEmpty@<#1>[#2]<\fdiff>{#3}}
\NewDocumentCommand{\UDerivativeEmpty}{d<> o m}{\DerivativeEmpty@<#1>[#2]<\udiff>{#3}}
\newcommand*\DrvE{\DerivativeEmpty}
\newcommand*\PDrvE{\PDerivativeEmpty}
\newcommand*\FDrvE{\FDerivativeEmpty}
\newcommand*\UDrvE{\UDerivativeEmpty}

% \DerivativeParentheses{what}{by-what} + shorthands for Partial, Finite and Delta
\NewDocumentCommand{\DerivativeParen}{d<> o m m}{\DerivativeParentheses@<#1>[#2]<\diff>{#3}{#4}}
\NewDocumentCommand{\PDerivativeParen}{d<> o m m}{\DerivativeParentheses@<#1>[#2]<\pdiff>{#3}{#4}}
\NewDocumentCommand{\FDerivativeParen}{d<> o m m}{\DerivativeParentheses@<#1>[#2]<\fdiff>{#3}{#4}}
\NewDocumentCommand{\UDerivativeParen}{d<> o m m}{\DerivativeParentheses@<#1>[#2]<\udiff>{#3}{#4}}
\newcommand*\DrvP{\DerivativeParen}
\newcommand*\PDrvP{\PDerivativeParen}
\newcommand*\FDrvP{\FDerivativeParen}
\newcommand*\UDrvP{\UDerivativeParen}

% \DerivativeEvaluate[order]{what}{by-what}{where}
% \DerivativeEval[2]{f(x)}{x}{0} is second derivative of f(x) at x = 0
\NewDocumentCommand{\DerivativeEval}{O{} m m m}{%
    \Eval{\Derivative[#1]{#2}{#3}}{#3 = #4}%
}%
\newcommand*\DrvEval{\DerivativeEval}

% \PartialDerivativeEvaluate[order]{what}{by-what}{where}
% \PDerivativeEval[2]{f(x)}{x}{0} is second derivative of f(x) at x = 0
\NewDocumentCommand{\PDerivativeEval}{O{} m m m}{%
    \EvalAt{\PDerivative[#1]{#2}{#3}}{#3 = #4}%
}%


% Generic 1D integral
% [lower limit]
% [upper limit]
% {integrand}
% {differential}
% [operation] between integrand and differential, such as \cdot or \times
\NewDocumentCommand{\Int@}{O{} O{} O{} m O{}}{\int\limits_{#1}^{#2} #3#5\diff#4}

% Basic integral from #1 to #2 of #3 with respect to #4
% Usage \Int[0][1]{x^2}{x}
\NewDocumentCommand{\Int}{O{} O{} m m}{\Int@[#1][#2][#3]{#4}[]}
\NewDocumentCommand{\IntP}{O{} O{} m m}{\Int@[#1][#2][\left(#3\right)]{#4}[]}
% Empty integral for operations on expressions
\NewDocumentCommand{\IntX}{O{} O{} m}{\int\limits_{#1}^{#2}{#3}}
\NewDocumentCommand{\IntE}{O{} O{} m}{\Int@[#1][#2][]{#3}[]}
% Integral from #1 to #2 of dot product of #3 and d#4
\NewDocumentCommand{\IntD}{O{} O{} m m}{\Int@[#1][#2][#3]{#4}[\cdot]}
% ...auto-vectorized version
\NewDocumentCommand{\IntDV}{O{} O{} m m}{\IntD[#1][#2]{\vec{#3}}{\vec{#4}}}
% Integral from #1 to #2 of cross product of #3 and d#4
\NewDocumentCommand{\IntC}{O{} O{} m m}{\Int@[#1][#2][#3]{#4}[\times]}
% ...auto-vectorized version
\NewDocumentCommand{\IntCV}{O{} O{} m m}{\IntC[#1][#2]{\vec{#3}}{\vec{#4}}}

% Loop integrals

% Generic loop integral
\NewDocumentCommand{\OInt@}{O{} m m d<>}{\oint\limits_{#1}#2#4\diff#3}
% Loop integral over #1
\NewDocumentCommand{\OInt}{O{} m m}{\OInt@[#1]{#2}{#3}<>}
% Loop integral over #1 of dot product of #2 and d#3 (for instance magnetic B dl)
\NewDocumentCommand{\OIntD}{O{} m m}{\OInt@[#1]{#2}{#3}<\cdot>}
\NewDocumentCommand{\OIntDV}{O{} m m}{\OIntD[#1]{\vec{#2}}{\vec{#3}}}
% Loop integral over #1 of cross product of #2 and d#3 (for instance magnetic B dl)
\NewDocumentCommand{\OIntC}{O{} m m}{\OInt@[#1]{#2}{#3}<\cross>}
\NewDocumentCommand{\OIntCV}{O{} m m}{\OIntC[#1]{\vec{#2}}{\vec{#3}}}

% Generic 2D integral
% [lower bound][upper bound]{integrand}{differential 1}{differential 2}
\NewDocumentCommand{\IInt}{O{} O{} m m m}{\iint\limits_{#1}^{#2}#3\diff#4\diff#5}
\NewDocumentCommand{\IIntP}{O{} O{} m m m}{\iint\limits_{#1}^{#2}\left(#3\right)\diff#4\diff#5}
% Surface integral over single differential
\NewDocumentCommand{\IIntI@}{O{} O{} m O{} m O{}}{\iint\limits_{#1}^{#2}#3#6\diff^{#4}#5}
\NewDocumentCommand{\IIntI}{O{} O{} m O{} m}{\IIntI@[#1][#2]{#3}{#4}}
% Surface integral of dot product over single differential
\NewDocumentCommand{\IIntD}{O{} O{} m m}{\IIntI@[#1][#2]{#3}{#4}[\cdot]}
\NewDocumentCommand{\IIntDV}{O{} O{} m m}{\IIntD[#1][#2]{\vec{#3}}{\vec{#4}}}
% Surface integral of cross product over single differential
\NewDocumentCommand{\IIntC}{O{} O{} m m}{\IIntI@[#1][#2]{#3}{#4}[\times]}
\NewDocumentCommand{\IIntCV}{O{} O{} m m}{\IIntC[#1][#2]{\vec{#3}}{\vec{#4}}}

% Surface integral over single differential
\NewDocumentCommand{\OIIntI@}{O{} O{} m m O{}}{\oiint\limits_{#1}^{#2}#3#5\diff#4}
\NewDocumentCommand{\OIIntI}{O{} O{} m m}{\OIIntI@[#1][#2]{#3}{#4}}
% Surface integral of dot product over single differential
\NewDocumentCommand{\OIIntD}{O{} O{} m m}{\OIIntI@[#1][#2]{#3}{#4}[\cdot]}
% Same, auto vectorize arguments
\NewDocumentCommand{\OIIntDV}{O{} O{} m m}{\OIIntD[#1][#2]{\vec{#3}}{\vec{#4}}}
% Surface integral of cross product over single differential
\NewDocumentCommand{\OIIntC}{O{} O{} m m}{\OIIntI@[#1][#2]{#3}{#4}[\times]}
% Same, auto vectorize arguments
\NewDocumentCommand{\OIIntCV}{O{} O{} m m}{\OIIntC[#1][#2]{\vec{#3}}{\vec{#4}}}

% Generic 3D integral

% [lower][upper]{integrand}{diff_1}{diff_2}{diff_3}
\NewDocumentCommand{\IIInt}{O{} O{} m m m m}{\iiint\limits_{#1}^{#2}#3\diff#4\diff#5\diff#6}
% Volume integral over single differential
% [lower limit][upper limit]{integrand}[differential power]{differential}
\NewDocumentCommand{\IIIntI@}{O{} O{} m O{} m O{}}{\iiint\limits_{#1}^{#2}#3#6\diff^{#4}#5}
\NewDocumentCommand{\IIIntI}{O{} O{} m O{} m}{\IIIntI@[#1][#2]{#3}{#5}[][#4]}
\NewDocumentCommand{\IIIntV}{O{} O{} m m}{\IIIntI@[#1][#2]{#3}[3]{#4}[]}
\NewDocumentCommand{\IIIntPV}{O{} O{} m m}{\IIIntI@[#1][#2]{\left(#3\right)}[3]{#4}[]}

\DeclareMathOperator{\del}{\raisebox{0.06em}{\ensuremath{\vec{\nabla}}}}

\NewDocumentCommand{\Laplacian}{}{\mathop{}\!\laplacian\!}

\DeclareMathOperator{\Grad}{\del\!}
\DeclareMathOperator{\GradT}{grad}
\NewDocumentCommand{\GradV}{m}{\Grad{\vec{#1}}}
\NewDocumentCommand{\GradTV}{m}{\GradT{\vec{#1}}}

\DeclareMathOperator{\Div}{\del\cdot}
\DeclareMathOperator{\DivT}{div}
\NewDocumentCommand{\DivV}{m}{\Div{\vec{#1}}}
\NewDocumentCommand{\DivTV}{m}{\DivT{\vec{#1}}}

% Rotation (with nabla)
\DeclareMathOperator{\Rot}{\del\times}
% Rotation (text)
\DeclareMathOperator{\RotT}{rot}

% Rotation (vectorize, nabla)
\NewDocumentCommand{\RotV}{m}{\Rot{\vec{#1}}}
% Rotation (vectorize, text)
\NewDocumentCommand{\RotTV}{m}{\RotT{\vec{#1}}}

\NewDocumentCommand{\ArrowVector}{m}{\vv{#1}}
\NewDocumentCommand{\LongVector}{m}{\overrightarrow{#1}}
\NewDocumentCommand{\BoldVector}{m}{\boldsymbol{#1}}
\NewDocumentCommand{\UnitVector}{m}{\hat{\vec{#1}}}
\NewDocumentCommand{\UnitBoldVector}{m}{\hat{\BoldVector{#1}}}
\NewDocumentCommand{\UnitArrowVector}{m}{\hat{#1}}

\def\omicron{o}
\DeclareMathOperator{\atantwo}{atan2}
\DeclareMathOperator{\arccot}{arccot}

% Logical operators
\DeclareMathOperator{\LXor}{\quad\lxor\quad}
\DeclareMathOperator{\LOr}{\quad\lor\quad}
\DeclareMathOperator{\LAnd}{\quad\land\quad}
\DeclareMathOperator{\LNand}{\quad\lnand\quad}

\DeclareMathOperator{\union}{\cup}
\DeclareMathOperator{\intersection}{\cap}

\NewDocumentCommand{\Implies}{}{\quad\Rightarrow\quad}
\NewDocumentCommand{\Iff}{}{\quad\Leftrightarrow\quad}

% Sinc
\DeclareMathOperator{\sinc}{sinc}
% Haversine
\DeclareMathOperator{\hav}{hav}

% Evaluation at point
\NewDocumentCommand{\Eval}{m m O{}}{\left.#1\right|_{#2}^{#3}}
\NewDocumentCommand{\EvalP}{m m O{}}{\left(#1\right)_{#2}^{#3}}
\NewDocumentCommand{\EvalB}{m m O{}}{\left[#1\right]_{#2}^{#3}}
% Expected value
\NewDocumentCommand{\ExpectedChevrons}{m}{\left<#1\right>}
\NewDocumentCommand{\ExpectedE}{m}{\mathrm{E}\left[#1\right]}
\NewDocumentCommand{\Mean}{m}{\overline{#1}}
\NewDocumentCommand{\Var}{m}{\mathrm{Var}\left(#1\right)}
\NewDocumentCommand{\MSE}{m}{\mathrm{MSE}\left({#1}\right)}
\NewDocumentCommand{\Bias}{m}{\mathrm{Bias}\left({#1}\right)}

% Text with spaces on both sides
\NewDocumentCommand{\QText}{m}{\quad\text{#1}\quad}
% Text with double spaces on both sides
\NewDocumentCommand{\QQText}{m}{\qquad\text{#1}\qquad}

\NewDocumentCommand{\Floor}{m}{\left\lfloor#1\right\rfloor}
\NewDocumentCommand{\Ceil}{m}{\left\lceil#1\right\rceil}

\NewDocumentCommand{\Lim}{m m}{\lim\limits_{#1 \rightarrow #2}}
\NewDocumentCommand{\Exp}{m}{e^{#1}}
\NewDocumentCommand{\LogTen}{}{\log_{10}}
\NewDocumentCommand{\Must}{m}{\stackrel{!}{#1}}
\NewDocumentCommand{\MustEqual}{}{\Must{=}}
\NewDocumentCommand{\DefEqual}{}{\stackrel{\mathrm{def}}{=}}

% Nuclide[A][Z]{symbol}
% \Nuclide[235][92]{U} for uranium
\NewDocumentCommand{\Nuclide}{O{} O{} m}{\ce{^{#1}_{#2}#3}}

% Big aggregate functions, and parenthesizing versions
\NewDocumentCommand{\Aggregate}{m O{} O{} m}{#1\limits_{#2}^{#3} #4}
\NewDocumentCommand{\AggregateP}{m O{} O{} m}{\Aggregate{#1}[#2][#3]{\left(#4\right)}}

% sum (big sigma)
% \Sum[lower limit][upper limit]{element}
\NewDocumentCommand{\Sum}{O{} O{} m}{\Aggregate{\sum}[#1][#2]{#3}}
\NewDocumentCommand{\SumP}{O{} O{} m}{\AggregateP{\sum}[#1][#2]{#3}}

% product (big pi)
% \Product[lower limit][upper limit]{element}
\NewDocumentCommand{\Product}{O{} O{} m}{\Aggregate{\prod}[#1][#2]{#3}}
\NewDocumentCommand{\ProductP}{O{} O{} m}{\AggregateP{\prod}[#1][#2]{#3}}

% cartesian product (big cross)
% \CartesianProduct[lower limit][upper limit]{element}
\NewDocumentCommand{\CartesianProduct}{O{} O{} m}{\Aggregate{\bigtimes}[#1][#2]{#3}}
\NewDocumentCommand{\CartesianProductP}{O{} O{} m}{\AggregateP{\bigtimes}#1][#2]{#3}}

% logical union (big cup)
% \Union[lower limit][upper limit]{element}
\NewDocumentCommand{\Union}{O{} O{} m}{\Aggregate{\bigcup}[#1][#2]{#3}}
\NewDocumentCommand{\UnionP}{O{} O{} m}{\AggregateP{\bigcup}[#1][#2]{#3}}

% intersection (big cap)
% \Intersection[lower limit][upper limit]{element}
\NewDocumentCommand{\Intersection}{O{} O{} m}{\Aggregate{\bigcap}[#1][#2]{#3}}
\NewDocumentCommand{\IntersectionP}{O{} O{} m}{\AggregateP{\bigcap}[#1][#2]{#3}}

\DeclarePairedDelimiter\abs@{\lvert}{\rvert}
\NewDocumentCommand{\Abs}{m}{\abs@*{#1}}

\DeclareMathOperator{\maxop}{max\ }
\DeclareMathOperator{\minop}{min\ }
\NewDocumentCommand{\Max}{O{} m}{\underset{#1}{\maxop}{#2}}
\NewDocumentCommand{\Min}{O{} m}{\underset{#1}{\minop}{#2}}

\NewDocumentCommand{\Angle}{}{\sphericalangle}

% Astronomical magnitudes
\NewDocumentCommand{\Mag}{m}{\ensuremath{{#1}^{\mathrm{m}}}}

% Distribution #1 of variable #2 with parameters #3
\NewDocumentCommand{\Distribution}{m m m}{#1(#2 \mid #3)}

% Asymptotic notations
\NewDocumentCommand{\SmallO}{m}{\mathcal{o}\left(#1\right)}
\NewDocumentCommand{\BigO}{m}{\mathcal{O}\left(#1\right)}
\NewDocumentCommand{\BigTheta}{m}{\mathcal{\Theta}\left(#1\right)}

% Number set
\NewDocumentCommand{\Set}{m O{} O{}}{%
    \left\{#1\right\}\ifblank{#2}{}{_{#2}}\ifblank{#3}{}{^{#3}}%
}

% Operation to be done on both side of an equation
\NewDocumentCommand{\Operation}{+m}{%
    & \qquad \bigg/ #1%
}

% Number sets
\NewDocumentCommand{\Natural}{}{\mathbb{N}}
\NewDocumentCommand{\NaturalZero}{}{\mathbb{N}_0}
\NewDocumentCommand{\Integer}{}{\mathbb{Z}}
\NewDocumentCommand{\Rational}{}{\mathbb{Q}}
\NewDocumentCommand{\Real}{}{\mathbb{R}}
\NewDocumentCommand{\RealPos}{}{\mathbb{R}^{+}}
\NewDocumentCommand{\RealNeg}{}{\mathbb{R}^{-}}
\NewDocumentCommand{\RealNonneg}{}{\mathbb{R}_{\geq 0}}
\NewDocumentCommand{\RealNonpos}{}{\mathbb{R}_{\leq 0}}
\NewDocumentCommand{\Complex}{}{\mathbb{C}}
\NewDocumentCommand{\Quaternions}{}{\mathbb{H}}

% Function utilities
\NewDocumentCommand{\Domain}{m}{\mathrm{dom}(#1)}

% Intervals
\NewDocumentCommand{\IntervalCC}{m m}{\left[#1; #2\right]}
\NewDocumentCommand{\IntervalCO}{m m}{\left[#1; #2\right)}
\NewDocumentCommand{\IntervalOC}{m m}{\left(#1; #2\right]}
\NewDocumentCommand{\IntervalOO}{m m}{\left(#1; #2\right)}

\NewDocumentCommand{\kth}{m}{\ensuremath{#1^{\text{th}}}}

\NewDocumentCommand{\Expected}{m}{\ExpectedE{#1}}

\let\tmp\phi
\let\phi\varphi
\let\varphi\tmp
\let\tmp\epsilon
\let\epsilon\varepsilon
\let\varepsilon\tmp

% Patch \left and \right with package mleftright
\let\left\mleft
\let\right\mright
