\DeclareMathOperator{\del}{\raisebox{0.06em}{\ensuremath{\vec{\nabla}}}}

\NewDocumentCommand{\Laplacian}{}{\mathop{}\!\laplacian\!}

\DeclareMathOperator{\Grad}{\del\!}
\DeclareMathOperator{\GradT}{grad}
\NewDocumentCommand{\GradV}{m}{\Grad{\vec{#1}}}
\NewDocumentCommand{\GradTV}{m}{\GradT{\vec{#1}}}

\DeclareMathOperator{\Div}{\del\cdot}
\DeclareMathOperator{\DivT}{div}
\NewDocumentCommand{\DivV}{m}{\Div{\vec{#1}}}
\NewDocumentCommand{\DivTV}{m}{\DivT{\vec{#1}}}

% Rotation (with nabla)
\DeclareMathOperator{\Rot}{\del\times}
% Rotation (text)
\DeclareMathOperator{\RotT}{rot}

% Rotation (vectorize, nabla)
\NewDocumentCommand{\RotV}{m}{\Rot{\vec{#1}}}
% Rotation (vectorize, text)
\NewDocumentCommand{\RotTV}{m}{\RotT{\vec{#1}}}

\NewDocumentCommand{\ArrowVector}{m}{\vv{#1}}
\NewDocumentCommand{\LongVector}{m}{\overrightarrow{#1}}
\NewDocumentCommand{\BoldVector}{m}{\boldsymbol{#1}}
\NewDocumentCommand{\UnitVector}{m}{\hat{\vec{#1}}}
\NewDocumentCommand{\UnitBoldVector}{m}{\hat{\BoldVector{#1}}}
\NewDocumentCommand{\UnitArrowVector}{m}{\hat{#1}}
